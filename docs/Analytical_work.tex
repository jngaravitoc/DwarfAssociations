\documentclass[12pt]{article}
\usepackage[hmargin=2.0cm,vmargin=1cm]{geometry}
\usepackage[utf8]{inputenc}
\usepackage{graphicx}
\usepackage{float}
\usepackage{cite}
\usepackage{natbib}
\usepackage{amsmath}

\title{\begin{LARGE}
{Analytical work}
\end{LARGE}}

\begin{document}

\maketitle

\section{Observed Associations}

\subsection{Associations density}

The observed associations have densities in the range: $[0.02-0.13 \dfrac{10^{11}M_{\odot}}{Mpc^3}]$
(Tully et.al 2006 table 2)

\subsection{Volume Observed}

The observations were in the range of, $r = [1.1-3.2 Mpc]$ and $b > |30|$

this means a volume of: 
\begin{equation}
V = \dfrac{-2\pi r^3 cos(\theta)}{3} \bigg|_{1.1}^{3.2} \bigg|_{\pi/6}^{5\pi/6} =  \dfrac{-2\pi 31.437 cos(\theta)}{3} \bigg|_{\pi/6}^{5\pi/6}
= \dfrac{-2 \pi 31.437 (-\sqrt{3})}{3} = 114.04 Mpc^{3}
\end{equation}


\subsection{Espected associations in sumulations?}

There where 7 associations in the volume computed before. Which leads to infere the expected number of associations in our volume:

\begin{equation}
\dfrac{7}{114.04} = \dfrac{3N_{sim}}{4\pi (7Mpc/h)^3} = \dfrac{N_sim}{4188.79 Mpc^3}
\end{equation} 

\begin{equation}
N_sim = \dfrac{7 \times 4188.79}{114.04} = 257.11 
\end{equation}

\end{document}
